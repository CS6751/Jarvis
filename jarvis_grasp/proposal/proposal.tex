\documentclass[10pt]{article}

\begin{document}
Alex Elhage

\section{Project Tasks}
My goal for this project is to complete a package which accomplishes the following tasks:
\begin{itemize}
\item Determine optimal pose of robot to allow for gripping circuit board
\item Detect a successful grasp of circuit board
\item Hand off the circuit board to user
\end{itemize}

\subsection{Pose Selection}
The pose determination node must be able to determine the optimum configuration for the system in order to grasp the circuit board. A list of possible locations to grasp will be provided by the perception modules. This node will then determine the optimal pose which allows the circuit board to be grasped. The exact definition of optimum will need to be determined, but this will likely include measurements of manipulability as well as considerations of safety near humans. This node only completes this function when the object is not currently being grasped, as determined by another node in this package.

\subsection{Grasp Detection}
As the robot moves towards the desired grasp, a node must be able to detect when a grasp is completed successfully. Upon a successful grasp, this node will publish a message to notify the executive of the change in state. This detection can be done using force sensors on gripper or by using the current to the motors as a relative measure of force being applied.

\subsection{Hand Off}
When the grasp is successful, this node will determine the goal pose in order to hand off the circuit board to the user. This will continually propose configurations while the object is being grasped. It is the responsibility of the consumer to determine from the global state when this information should be used.

\section{Research Proposal}
\subsection{Determination of Optimal Pose}
A manipulability factor will be developed which accounts not only for the manipulability ellipsoid of the manipulator, but also accounts for static obstacles in the environment as well as a probabilistic model of where the human is likely to move. The original measure of manipulability concerned itself with limitations of motion due to the kinematics of the manipulator and joint velocity limits \cite{Yoshikawa1985}. This measure has been expanded to other factors such as joint position limits and static obstacles \cite{Vahrenkamp2012}. In this project, however, the manipulator will be working closely with a dynamic obstacle, i.e., the human. Incorporating a model of where the user is likely to be based on the user's current pose and the environment will allow for selection of a pose that is likely to not be an obstacle to the user's actions in the future.

This portion of the research is of low priority because it has minimal benefits for this project. The manipulator being used will be fixed to a desk and has limited redundancy. As a result, it is likely that in many scenarios it is only possible to generate one candidate pose, meaning the manipulability factor will not be of use. Time permitting, however, this portion can be developed and demonstrated using the mobile base for the manipulator.

\subsection{Hand Off}


\section{Development Tasks}
Tasks which are in bold are critical to the function of the system. Tasks in plain style are research tasks which are not critical, but are the goal of this project. Tasks in italic are extra work to be completed only if all other tasks are completed.

\begin{itemize}
\item Pose Selection
  \begin{itemize}
  \item \textbf{Create a node which performs inverse kinematics for provided grasp points and selects any potential pose for grasping}
  \item \textit{Develop manipulability factor which incorporates probabilistic model for user's motion}
  \item \textit{Develop model for user's motion based on their current pose}
  \end{itemize}
\item Grasp Detection
  \begin{itemize}
  \item \textbf{Develop library which communicates with motor controllers and models torque-current curves to determine torques at end effector} (will also be critical for human intent package and ``Hand Off'' task)
  \item Determine threshold for gripper torque that is considered successful grasp
  \end{itemize}
\item Hand Off
  \begin{itemize}
  \item \textbf{Develop node which can switch the current mode and release the object being gripped}
  \item Collect data for end-effector force during hand offs
  \item Develop model for predicting grasp type based on end-effector force
  \item Determine optimum orientation for hand off based on grasp type
  \item Implement controller to move towards optimum orientation after grasp type is detected
  \end{itemize}
\end{itemize}

\bibliographystyle{plain}
\bibliography{references}
\end{document}
